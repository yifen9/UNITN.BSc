\documentclass[12pt]{article}

\usepackage{graphicx}
\usepackage{amssymb}
\usepackage{amsmath}
\usepackage[margin=1in]{geometry}
\usepackage{fancyhdr}
\usepackage{enumerate}
\usepackage[shortlabels]{enumitem}

\pagestyle{fancy}
\fancyhead[l]{Li Yifeng}
\fancyhead[c]{Homework \#4}
\fancyhead[r]{\today}
\fancyfoot[c]{\thepage}
\renewcommand{\headrulewidth}{0.2pt}
\setlength{\headheight}{15pt}

\newcommand{\bP}{\mathbb{P}}

\begin{document}
	
	\section*{Question 4}
	
	\noindent Consider a routine screening test for a disease. Suppose the frequency of the disease in the population (base rate) is $0.5\%$. The test is fairly accurate with a $5\%$ false positive rate and a $10\%$ false negative rate. You take the test and it comes back positive. What is the probability that you have the disease?
	
	\bigskip
	
	\begin{enumerate}[label={},leftmargin=0in]\item
		
		\subsection*{Solution}
		
			Let $\it{positive}$ and $\it{negative}$ denote the results of a test. Define the probability space as
			
			\[
			\begin{aligned}
				\Omega &= \{\mathrm{positive},\,\mathrm{negative}\}\\
				\mathcal{F} &= \mathcal{P}(\Omega)\\
				\bP &:\enspace \text{the probability measure on $\mathcal{F}$}
			\end{aligned}
			\]
			
			Let $B$ be the event that having the disease in the population (base rate)
			
			\[\bP(B) = 0.5\%\]
			
			Then we have the false positive rate ($\bP_{B^c}(\{\mathrm{positive}\})$) and the false negative rate ($\bP_B(\{\mathrm{negative}\})$)
			
			\[
			\begin{aligned}
				\bP_{B^c}(\{\mathrm{positive}\}) &= 5\%\\
				\bP_B(\{\mathrm{negative}\}) &= 10\%
			\end{aligned}
			\]
			
			Then we have the true negative rate ($\bP_{B^c}(\{\mathrm{negative}\})$) and the true positive rate ($\bP_B(\{\mathrm{positive}\})$)
			
			\[
			\begin{aligned}
				\bP_{B^c}(\{\mathrm{negative}\}) &= 1-\bP_{B^c}(\{\mathrm{positive}\}) &= 95\%\\
				\bP_B(\{\mathrm{positive}\}) &= 1-\bP_B(\{\mathrm{negative}\}) &= 90\%
			\end{aligned}
			\]
			
			Then by applying the law of total probability, we have
			
			\[\bP(\{\mathrm{positive}\}) = \bP(B)\bP_B(\{\mathrm{positive}\}) + \bP(B^c)\bP_{B^c}(\{\mathrm{positive}\}) = 5.425\%\]
			
			By applying the Bayes’ Theorem, we have
			
			\[
			\bP_{\{\mathrm{positive}\}}(B) = \frac{\bP_B(\{\mathrm{positive}\})\bP(B)}{\bP(\{\mathrm{positive}\})} = \frac{18}{217} \approx 8.295\%
			\]
		
		\subsection*{Answer}
		
			\[\boxed{\bP_{\{\mathrm{positive}\}}(B) = \frac{18}{217} \approx 8.295\%}\]

	\end{enumerate}
	
\end{document}
