\documentclass[12pt]{article}

\usepackage{graphicx}
\usepackage{amssymb}
\usepackage{amsmath}
\usepackage[margin=1in]{geometry}
\usepackage{fancyhdr}
\usepackage{enumerate}
\usepackage[shortlabels]{enumitem}

\pagestyle{fancy}
\fancyhead[l]{Li Yifeng}
\fancyhead[c]{Homework \#9}
\fancyhead[r]{\today}
\fancyfoot[c]{\thepage}
\renewcommand{\headrulewidth}{0.2pt}
\setlength{\headheight}{15pt}

\newcommand{\bE}{\mathbb{E}}
\newcommand{\bP}{\mathbb{P}}

\begin{document}
	
	\section*{Question 2}
	
	\noindent Suppose that women who live beyond the age of $80$ outnumber men in the same age group by three to one. How much information, in bits, is gained by learning that a person who lives beyond $80$ is male?
	
	\bigskip
	
	\begin{enumerate}[label={},leftmargin=0in]\item
		\subsection*{Solution}
		
			Let
			
			\[
				\begin{aligned}
					\bP(W) &:= \text{probability of being a woman beyond the age of $80$}\\
					\bP(M) &:= \text{probability of being a man beyond the age of $80$}
				\end{aligned}
			\]
			
			Then we have
			
			\[
				\begin{aligned}
					\bP(W) &= \frac{3}{4}\\
					\bP(M) &= \frac{1}{4}
				\end{aligned}
			\]
			
			So that
			
			\[
				I(X) = -log_2(\bP(M)) = 2 \text{bits}
			\]

		\subsection*{Answer}
		
		\[\boxed{I(X) = 2 \text{bits}}\]
	\end{enumerate}
	
\end{document}