\documentclass[12pt]{article}

\usepackage{graphicx}
\usepackage{amssymb}
\usepackage{amsmath}
\usepackage[margin=1in]{geometry}
\usepackage{fancyhdr}
\usepackage{enumerate}
\usepackage[shortlabels]{enumitem}

\pagestyle{fancy}
\fancyhead[l]{Li Yifeng}
\fancyhead[c]{Homework \#10}
\fancyhead[r]{\today}
\fancyfoot[c]{\thepage}
\renewcommand{\headrulewidth}{0.2pt}
\setlength{\headheight}{15pt}

\newcommand{\bE}{\mathbb{E}}
\newcommand{\bP}{\mathbb{P}}

\begin{document}
	
	\section*{Question 4}
	
	\noindent Consider a binary communication channel with input $X\sim\mathrm{Bern}(p)$, output $Y\sim\mathrm{Bern}(q)$, and crossover (error) probability $\varepsilon$.  It is known that
	\[
	q \;=\; p + \varepsilon \;-\; 2\varepsilon p.
	\]
	The channel capacity is
	\[
	C \;=\;\max_{0\le p\le1} I(X;Y).
	\]
	Find the value of $p$ that maximises $C$ (i.e.\ maximises $I(X;Y)$).
	
	\bigskip
	
	\begin{enumerate}[label={},leftmargin=0in]\item
		\subsection*{Solution}
		The mutual information can be written
		\[
		I(X;Y)
		=H(Y)-H(Y\mid X)
		=H\bigl(q\bigr)-H\bigl(\varepsilon\bigr),
		\]
		where $H(\varepsilon)=-\varepsilon\log_2\varepsilon-(1-\varepsilon)\log_2(1-\varepsilon)$ is constant in $p$, and
		\[
		q(p)=p + \varepsilon -2\varepsilon p
		=(1-2\varepsilon)p + \varepsilon.
		\]
		Differentiate with respect to $p$:
		\[
		\frac{dI}{dp}
		=\frac{d}{dp}H(q)
		=(1-2\varepsilon)\,\bigl[-\log_2 q + \log_2(1-q)\bigr]
		\stackrel{!}{=}0.
		\]
		Hence
		\[
		\log_2\frac{1-q}{q}=0
		\;\Longrightarrow\;q=\tfrac12.
		\]
		Solving $q=(1-2\varepsilon)p+\varepsilon=\tfrac12$ gives
		\[
		(1-2\varepsilon)p=\tfrac12-\varepsilon
		\;\Longrightarrow\;
		p=\frac{\tfrac12-\varepsilon}{1-2\varepsilon}
		=\tfrac12,
		\]
		for all $\varepsilon\neq\tfrac12$.
		
		\subsection*{Answer}
		\[
		\boxed{p=\frac12.}
		\]
	\end{enumerate}
	
\end{document}