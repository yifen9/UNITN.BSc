\documentclass[12pt]{article}
\usepackage{amsmath, amssymb}
\usepackage{geometry}
\geometry{a4paper, margin=1in}
\usepackage{setspace}
\usepackage{lmodern}
\usepackage{titlesec}

% Formatting section headings
\titleformat{\section}[block]{\bfseries\Large}{\thesection.}{1em}{}
\titleformat{\subsection}[block]{\bfseries\large}{\thesubsection.}{1em}{}

\title{Question 1: Statistical Analysis of a Sample Population}
\author{}
\date{}

\begin{document}
	\maketitle
	\onehalfspacing
	
	\section*{Problem Statement}
	A sample consists of 120 males and 80 females. The table below shows the distribution of individuals by gender and age.
	
	\bigskip
	
	\section*{Part 1: Total Number of Individuals Younger Than 20}
	Let the number of males ($n_{\mathrm{male}}$) and females ($n_{\mathrm{female}}$) in the sample be:
	\[
	\begin{aligned}
		n_{\mathrm{male}} &= 120, \\
		n_{\mathrm{female}} &= 80.
	\end{aligned}
	\]
	Thus, the total sample size is:
	\[
	n = n_{\mathrm{male}} + n_{\mathrm{female}} = 200.
	\]
	Let the number of age groups be:
	\[
	K = 4.
	\]
	Define the relative frequencies of males ($p_{\mathrm{male}_k}$) and females ($p_{\mathrm{female}_k}$) in each group:
	\[
	\begin{alignedat}{4}
		p_{\mathrm{male}_1} &= 10\%,\quad p_{\mathrm{male}_2} &= 10\%,\quad \dots,\quad p_{\mathrm{male}_K} &= 50\%, \\
		p_{\mathrm{female}_1} &= 20\%,\quad p_{\mathrm{female}_2} &= 20\%,\quad \dots,\quad p_{\mathrm{female}_K} &= 30\%.
	\end{alignedat}
	\]
	The absolute frequency of males and females in each group is given by:
	\[
	\begin{alignedat}{2}
		N_{\mathrm{male}_k} &= n_{\mathrm{male}} p_{\mathrm{male}_k},\quad &k \in \{1,2,\dots,K\}, \\
		N_{\mathrm{female}_k} &= n_{\mathrm{female}} p_{\mathrm{female}_k},\quad &k \in \{1,2,\dots,K\}.
	\end{alignedat}
	\]
	The total absolute frequency is:
	\[
	N_k = N_{\mathrm{male}_k} + N_{\mathrm{female}_k},\quad k \in \{1,2,\dots,K\}.
	\]
	The number of people younger than 20 is:
	\[
	N_{\#\{y/o \in \{0,1,\dots,19\}\}} = N_1 = 28.
	\]
	\textbf{Answer for Part 1:}
	\[
	\boxed{N_{\#\{y/o \in \{0,1,\dots,19\}\}} = 28}.
	\]
	
	\bigskip
	
	\section*{Part 2: Percentage of Individuals Aged 50 or Older}
	The relative frequency for the total population is:
	\[
	p_k = \frac{N_k}{n},\quad k \in \{1,2,\dots,K\}.
	\]
	The percentage of individuals aged 50 or older is:
	\[
	p_{\frac{\#\{y/o \in \{50,51,\dots,89\}\}}{\#\{y/o \in \{0,1,\dots,89\}\}}} = p_4 = 42\%.
	\]
	\textbf{Answer for Part 2:}
	\[
	\boxed{p_{\frac{\#\{y/o \in \{50,51,\dots,89\}\}}{\#\{y/o \in \{0,1,\dots,89\}\}}} = 42\%}.
	\]
	
	\bigskip
	
	\section*{Part 3: Number of Males Aged 30 or Older}
	The number of males aged 30 years or older is:
	\[
	N_{\mathrm{male}_{\#\{y/o \in \{30,31,\dots,89\}\}}} = \sum_{k=3}^{K} N_{\mathrm{male}_k} = 96.
	\]
	\textbf{Answer for Part 3:}
	\[
	\boxed{N_{\mathrm{male}_{\#\{y/o \in \{30,31,\dots,89\}\}}} = 96}.
	\]
	
	\bigskip
	
	\section*{Part 4: Median Age Calculation}
	Define the age class intervals:
	\[
	\begin{aligned}
		z_1 &= y/o \in \{0,1,\dots,19\}, \\
		z_2 &= y/o \in \{20,21,\dots,29\}, \\
		&\dots \\
		z_K &= y/o \in \{50,51,\dots,89\},\quad k \in \{1,2,\dots,K\}.
	\end{aligned}
	\]
	The median position is at:
	\[
	p_m = 50\%.
	\]
	Since:
	\[
	\sum_{k=1}^{2} p_k = 28\% < p_m < 58\% = \sum_{k=1}^{3} p_k,
	\]
	the median falls within $z_3$. The lower boundary of $z_3$ is:
	\[
	L_{z_3} = \frac{29+30}{2} = 29.5.
	\]
	The class width is:
	\[
	C_{z_3} = \frac{(49+50)-(29+30)}{2} = 20.
	\]
	Assuming a uniform distribution within $z_3$, the median is calculated as:
	\[
	m = L_{z_3} + \left(\frac{p_m - \sum_{k=1}^{2} p_k}{p_3}\right) C_{z_3}.
	\]
	Substituting values:
	\[
	m = 29.5 + \left(\frac{50\% - 28\%}{30\%}\right) \times 20 = \frac{265}{6} \approx 44.167.
	\]
	\textbf{Answer for Part 4:}
	\[
	\boxed{m = \frac{265}{6} \approx 44.167}.
	\]
	
\end{document}
